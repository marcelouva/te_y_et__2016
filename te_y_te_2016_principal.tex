\documentclass{llncs}

\usepackage[utf8x]{inputenc}
\usepackage{amssymb}
\usepackage{amsmath}
\usepackage{graphicx}
\usepackage{multicol} 
 \usepackage[spanish]{babel} % espanol, ingles



% \documentclass{article}
%opening
\title{De la Universidad a la industria del software y viceversa: Una experiencia centrada en la aplicación de nuevas tecnologías, adoptadas por la industria, a proyectos finales de asignaturas}

\author{Marcela Daniele, Marcelo Uva, Ariel Arsaute y Franco Brusatti }

\institute{Departamento de Computaci\'on, FCEFQyN, Universidad Nacional de R\'{\i}o Cuarto, R\'{\i}o Cuarto, Argentina. 
Email: \email{$\{$marcela,uva,fbrusatti,aarsaute$\}$@dc.exa.unrc.edu.ar}
}

\begin{document}

 

\maketitle


\begin{abstract}
Durante la última década, la relación Universidad e industria de desarrollo de software se ha vuelto cada vez más estrecha. 
Pequeñas, medianas y grandes empresas se acercan continuamente a las universidades en busca de recursos humanos calificados.
El crecimiento y expansión del mercado informático ha propiciado la generación y adopción de nuevas 
metodologías de desarrollo, nuevas modalidades de trabajo como desarrollos outsourcing y freelance.
Al mismo tiempo, han surgido nuevas tecnologías que brindan el soporte necesario para las actividades de administración, 
gestión, planificación, implementación, diseño, prueba  y control en  la producción 
del software. 
En este trabajo se presenta una experiencia desarrollada durante los años 2014 y 2015 en el marco de las asignaturas de Ingeniería
de Sofware. La propuesta consiste en el desarrollo de un proyecto integrador que aplica un 
conjunto de herramientas y tecnologías de gran impacto en la industria del desarrollo de software actual, estableciendo un puente directo  
entre formación académica y necesidad del mercado.\\ \\
\textbf{Palabras Claves}: Ingeniería de Software, Análisis y Diseño de Sistemas, Industria del Software y  Scrum. 

\end{abstract}


\begin{multicols}{2} 



\section{Introducción}

Durante la última década, la relación Universidad-Industria de desarrollo de software se ha vuelto cada vez más estrecha. Pequeñas, medianas y grandes empresas se acercan a menudo a las universidades en busca de analistas, programadores, ingenieros de software, etc.
El constante crecimiento y expansión del mercado informático, en todos los ámbitos, hace cada vez más visible la necesidad de contar con 
recursos humanos calificados. Paralelamente, dicho crecimiento ha propiciado en la industria, la generación y adopción de nuevas 
metodologías de desarrollo y nuevas modalidades de trabajo por ej. desarrollos outsourcing.
Junto con éstas, han surgido nuevas tecnologías brindando el soporte necesario para las actividades de administración, 
gestión, planificación, implementación, diseño, automatización del testing, seguimiento y control de las tareas que hacen a la producción 
del software, entre otras. Muchas de éstas tecnologías poco a poco han ido  incorporandose a las currículas de las carreras de informática. De esta manera se ha establecido una relación de necesidad entre Universidad e Indrustria de desarrollo de sofware en ambos sentidos.
 
 
En el marco de los proyectos PIIMEG CITA (Proyectos de Investigación e Innovación para el Mejoramiento de la Enseñanza de Grado) pertenecientes a la Universidad
 Nacional de Río Cuarto se han realizado una serie de propuestas en pos de detectar, analizar y ejecutar acciones concretas con el 
 fin de realizar aportes para solucionar problemáticas observadas en las asignaturas de 3er. año de las carreras de Analista, Profesorado y 
 Licenciatura en Ciencias de la Computación. Algunas de las acciones realizadas se resumen a continuación:
 \begin{itemize}
 \item Desarrollo e un Proyecto-taller de integración de todas las asignaturas de 3er. año (Análisis y Diseño de Sistemas, Base de Datos y 
  Diseño de algoritmos. El problema observado en este caso radicaba en que los alumnos debían realizar un proyecto final  para cada una de las asignaturas
 de 3er. año. En algunos casos, la carga horaria de la asignatura tenía contemplado esto y en otras no. Esto producía que  alumnos
quedaran libres en algunas materias por no poder cumplir con los requisitos de las mismas. O bien, ellos mismos optaran por hacer unas u otras, debido a la exigencia horaria requerida.  
Para evitar todo lo anterior, se diseño un proyecto-taller integrador único para todas las asignaturas de 3er. año. Cada asignatura incluiría en éste 
los contenidos a evaluar que requiriera. El proyecto-taller consistiría en el desarrollo de un
 sistema informático en donde la asignatura Base de Datos se encargaría de diseñar y generar la base de datos.
 La asignatura Diseño de Algoritmos 
 debía establecer 
 algún requerimiento que implicara la necesidad de la utilización de técnicas algorítmicas estudiadas durante la cursada y, finalmente  Análisis y Diseño de Sistemas
 se encargaría del seguimiento y control del proceso de desarrollo. El dominio del proyecto sería definido por los cuerpos docentes de las tres asignaturas.CITA
 \item Definición de una plantilla para guiar la documentación en lo que respecta a proyectos desarrollados con metodologías ágiles.CITA 
\item Estudio y análisis de causas de retraso de la finalización de los Trabajos Finales CITA. 
\item Propuesta para realizar el Trabajo Final en los tiempos planificados CITA.
\end{itemize}

En este trabajo se presenta una experiencia desarrollada durante los años 2014 y 2015 dentro de las asignaturas Análisis y Diseño de Sistemas e Ingeniería de Software. 
La propuesta consiste en el desarrollo de un proyecto-taller en donde, además de integrar los contenidos teórico-prácticos abordados en las 
asignaturas de tercer año, se incorporen un conjunto de tecnologías utilizadas en la industria del desarrollo de software actual. \\



El resto del trabajo está organizado de la siguiente manera: en la sección 2 se presentan los fundamentos de la propuesta y se plantean los 
objetivos. La sección 3 presenta la metodología de desarrollo aplicada. En la sección 4 se presenta la propuesta, planificación, 
tecnologías y herramientas utlizadas. En la sección 5 se realiza una evalución de la propuesta, y finalmente, las conclusiones.

objetivos. La sección 3 presenta la metodología de desarrollo aplicada. En la sección 4 se presenta la propuesta, su planificación y 
las tecnologías y herramientas utlizadas. En la sección 5 se realiza una evalución de la propuesta, y finalmente, las conclusiones.
\section{Fundamentos}\label{fundamenta}
La planificación y ejecución de procesos de enseñanza-aprendizaje para cursos de ingeniería de software (IS), plantean un gran desafío a los docentes universitarios involucrados. La necesidad de una actualización dinámica de los contenidos no debe provocar el descuido de conceptos básicos vinculados a los principios fundamentales del desarrollo de sistemas de software.
El cuerpo docente, autor de este artículo, tiene a cargo el dictado de los cursos Análisis y Diseño de Sistemas e Ingeniería de Software, durante el tercer año de las carreras: Licenciatura y Profesorado en Ciencias de la Computación y Analista en Computación, de la Universidad Nacional de Río Cuarto.

Los conceptos básicos de IS, son introducidos en el curso de Análisis y Diseño de Sistemas. En este se presentan diferentes metodologías de desarrollo de software [ver$$$$]. 

Para el curso de Ingeniería de Software, el principal propósito es que el alumno tome conocimiento de los conceptos más avanzados de IS; desde la planificación y gestión del proyecto hasta técnicas de testing o prueba. Al mismo tiempo se cubren  conceptos transversales a las etapas de desarrollo como el gerenciamiento de la configuración de software



Esta metodología incorpora la realización de proyectos-taller, abordando específicamente el desafío de la utilización de herramientas que asistan en las actividades de gestión de proyectos de software [ver$$$$] y ayuden al mismo tiempo la compresión profunda de los temas abordados. 
El principal propósito es conseguir que los alumnos vivencien situaciones muy cercanas a la realidad y, de esta manera, disminuir la brecha entre la teoría universitaria y la realidad profesional.



El dictado de la asignatura IS se divide en clases teóricas, clases prácticas y proyectos-taller

Durante los años 2014 y 2015, se abordó importantes desafíos a cubrir en la enseñanza de ingeniería de software, introduciendo la utilización de herramientas que permitan automatizar la gestión de proyectos de software [Ver$$$$]. 

En este marco, se logró fomentar el uso de una metodología de desarrollo de software en proyectos reales. Promoviendo de esta manera la división de roles operativos y gerenciales entre los integrantes de un grupo de desarrollo de software, capacitándolos para la evaluación y selección de diferentes herramientas aplicables a las actividades de gestión de software.

En el año 2014, se presentó ……

En el año 2015, se presentó ……


\section{Metodología de desarrollo aplicada a la propuesta} \label{metodologia}
% \section{Metodologías Tradicionales y Metodologías Ágiles}\label{metodologia}
En el marco de las asignaturas Análisis y Diseño de Sistemas, perteneciente al tercer año de las carreras Analista, Profesorado y 
 Licenciatura en Ciencias de la Computación de la U.N.R.C, se presentan diferentes metodologías para el desarrollo de 
sistemas informáticos. Algunos de los ciclos de vida del software estudiados son:  el ciclo de vida lineal o secuencial, el método del análisis estructurado de Jourdon\cite{}
, desarrollo basado en componentes, diseño por contratos\cite{} de Bertrand Meyer, el Proceso Unificado \cite{} y metodologías ágiles.\\ 
 
La asignatura Análisis y Diseño de Sistemas cuenta con una planificación de 180 horas. Como se explicó en la sección FUNDAMENTOS, 
en trabajos ateriores se propuso
integrar este proyecto a los proyectos de fin de materia de todas las asignaturas de 3er. año. En esta oportunidad, se presenta 
una modificación a este proyecto-taller meditante la incorporación de un set de herramientas 
que la industria de desarrollo de software utiliza actualmente. Dichas tecnologías están intimamente ligadas a la aplicación de metodologías ágiles. 
Es por ello que el cuerpo docente de la asignatura ha seleccionado la metodología ágil SCRUM  CITA  la cual se explica brevemente en el siguiente apartado.

\subsection{Scrum}
Scrum es una metodología ágil que permite trabajar en ambientes muy cambiantes, permitiendo permanentes replanteamientos. 
Por otro lado permite reducir el tiempo de producción y de comercialización del producto obtenido, aporta un gran beneficio o 
valor agregado al cliente, minimiza los riesgos de desperdiciar esfuerzo/tiempo en la construcción de artefactos que no serán usados 
o que no son fundamentales para el cliente. Facilita también la comunicación entre todos los integrantes del proyecto. 
La documentación producida dentro de un proyecto SCRUM es relativa al usuario, dueño, producto y equipo.  Scrum es más bien un marco de 
trabajo que reposa sobre la premisa de que el equipo de desarrollo conocerá la mejor manera de resolver el problema que se le presenta. 
La reunión de planificación de cada grupo de requerimientos a producir se describe en términos del resultado deseado, 
en lugar de un conjunto de criterios de ingreso, definiciones y tareas. 
Scrum se basa en una auto-organización, con un equipo multifuncional y sin líder (dentro del equipo). 
El equipo es apoyado por dos individuos quienes ocupan los roles de Scrum Master y  Product Owner. 
El Scrum Master es una especie de entrenador para el equipo, su función es ayudar a los miembros del mismo a utilizar el marco que ofrece 
la metodología para conseguir un nivel alto de productividad. Mientras que el Product Owner representa los negocios, clientes o 
usuarios. Éste guía al equipo hacia la construcción del producto adecuado.
 Los proyectos Scrum avanzan en orden a la definición de los sprints, que son las iteraciones que poseen una duración de entre dos y cuatro semanas. 
En el inicio de cada sprint, los miembros del equipo se comprometen a producir un cierto número de características que enumeran en 
el artefacto conocido como el Product Backlog del proyecto.  Al final de cada sprint, cada funcionalidad (conocida como \emph{historia}) debe estar codificada,
 probada e integrada a una versión demo del sprint anterior. Luego se realiza una revisión, y por último se demuestra la nueva funcionalidad
 frente al Product Owner y las otras partes interesadas que proporcionen información requerida para el siguiente sprint. 
Las iteraciones han de continuar hasta obtener el producto deseado.
Como se puede observar de lo expuesto, SCRUM establece una forma de trabajo en donde todo el equipo se auto-regula y en donde no hay de antemano
una documentación establecida a priori. Cada equipo utilizará los elementos que le sean necesarios  para poder llevar adelante el grupo de \emph{historias}
con el cual se ha comprometido. Algunos podrán utilizar diagramas UML\cite{} otros utilizarán diagramas de flujos de datos, etc.
En la figura \ref{figu1} se esquematiza todo el proceso.
% \includegraphics[width=0.7 \textwidth]{f2}
% \begin{figure}
%   \centering
%     \includegraphics[width=0.5 \textwidth]{f2}
%   \caption{Proceso Scrum}
%   \label{figu1}
% \end{figure}

\end{multicols}

  \begin{figure}
   \centering
\includegraphics[width=0.9 \textwidth]{f2}
    \caption{Proceso de desarrollo Scrum}
    \label{figu1}
  \end{figure}



\begin{multicols}{2} 

\section{Descripción de la propuesta} \label{propuesta}

En este trabajo se presenta una experiencia desarrollada durante los años 2014 y 2015 dentro de la asignatura Análisis y Diseño de
Sistemas perteneciente a las carreras de computación dictadas dentro de la Universidad Nacional de Río Cuarto.
La propuesta consiste 
en el desarrollo de un proyecto-taller en donde, además de integrar los contenidos teórico-prácticos 
estudiados, se incorpore un conjunto de tecnologías y herramientas utilizadas en la industria del desarrollo de software actual,
permitiendo
establecer un mejor
puente entre el futuro egresado y las empresas de desarrollo de software. \\


\subsection{Objetivos de la propuesta}
Los objetivos inciales de la propuesta fueron los siguientes:
 \begin{itemize}
 \item Integrar en un  proyecto de desarrollo de software los contenidos estudiados en las asignaturas pertenecientes al tercer 
   año de las carreras de Analista, Profesorado y  Licenciatura en Ciencias de la Computación. La planificación CITA del proyecto esta 
   fuertemente articulada con la planificación de la asignatura.
 \item Aplicar una metodología de desarrollo ágil como Scrum. Como se explicó en la sección anterior es muy beneficiosa y es la que actualmente utilizan las grandes empresas de sofware.
 \item Simular un ambiente de trabajo con características similares a las de un ambiente de trabajo real. Para ello se establecen objetivos grupales y es el mismo grupo quien se auto-gestiona. Si bien los objetivos son grupales, en el marco del taller, se realiza un seguimiento particular de cada uno de los alumnos. Su integración con el resto, su compromiso, su forma de trabajo, etc.
 \item Desarrollar en los alumnos habilidades para adoptar y sacar provecho de un set de herramientas utilizadas en la industria del software. El objetivo no es directamente que aprendan a usar una herramienta, sino que sean capaces de adoptar esta u otra que pudiera surgir en un futuro.
 
 \end{itemize}
 
\subsection{Organización y planificación}
El taller es planificado para ejecutarse durante el primer cuatrimestre del año. Cabe señalar
que dentro del grupo de docentes de la asignatura, se ha seleccionado a un docente  con dedicación part-time para la coordinación general 
del taller. La selección de este docente fue motivada por la vinculación que  el mismo posee con la industria de desarrollo de software. 
Este docente es fundador de una empresa de desarrollo de software local y está habituado a formar parte de grupos de trabajo
con desarrolladores a nivel internacional.
Uno de los objetivos del taller es la aplicación de una metodología ágil, es por ello que el proyecto estará guiado por Srum.
En cada una de las fases del ciclo de vida, se aplicará una herramienta específica detallada en la sección CITA. No es objetivo de este 
proyecto complejizar ni sobrecargar tiempos de desarrollo. Sino que se focaliza en que cada 
grupo pueda completar el trabajo, usando las tecnologías propuestas.  
Los alumnos se agruparon en equipos de tres, más el Scrum Master CITA SCRUM, rol ocupado por un docente de la materia.\\
El proyecto consistió en la construcción de un  sistema web en donde un usuario  pueda loguearse y publicar avisos de compra
 y venta de vehículos. El diseño del problema se planteó de una manera simplificada para evitar que el proyecto creciera de 
 forma descontrolada. El cuerpo docente definió un Backlog inicial CITA SCRUM. y en la medida de que cada grupo avanzaba se incluían 
 nuevas funcionalidades. Todos los grupos tenían una clase/consulta semanal de dos horas en la que se presentaba el taller,
 se introducían las herramientas y se establecían los objetivos según la planificación. En el resto de las clases, los docentes realizaban el seguimiento de cada grupo.
 

A continuación se presenta la planificación para el proyecto-integrador
llevado a cabo durante el año 2015.
\begin{itemize}
 \item Marzo 18 - Presentación general del taller e introducción de Git.
\item Abril 1 - SCRUM y Definnición del SRS (Sofware Requeriment Specification). PivotalTracker - Creación del Backlog.
\item Abril 8 - Introducción a Java - Eclipse / ActiveJDBC / Postgresql.
\item Abril 15 - Taller de  Maven y JUnit.
\item Abril 22 - Diseño / Implementación.
\item Abril 29 - Diseño / Implementación.
\item Mayo 6 - Taller de JUnit - definición de la Test Suite.
\item Mayo 13 - Diseño / Implementación.
\item Mayo 20 - Checkpoint (Test Suite running).
\item Mayo 27 - Taller Spark - Sinatra e incorporación de la capa web.
\item Junio 3 - Implementación 
\item Junio 10 - Implementación
\item Junio 17 - Presentación de grupos
\end{itemize}

\subsection{Tecnologías y herramientas utilizadas}
La industria del software se encuentra en continua expansión. Esto requiere la incorporación de nuevas tecnologías para generar
nuevas tecnologías. Sólo incorporando nuevas herramientas de soporte para estos procesos es que se logra producir sistemas para millones
de usuarios como por ejemplo las redes sociales como Facebook o Twitter. \\
Las herramientas utilizadas en este proyecto fueron las siguientes:
 \begin{itemize}
 \item Github:CITA Es una plataforma de desarrollo colaborativo de software para alojar proyectos utilizando el sistema de control de versiones Git [14]. El código se almacena de forma pública, aunque también se puede hacer de forma privada, creando una cuenta de pago. 
 \item Git:CITA Es un sistema de control de versiones distribuido, libre y gratuito.
 \item Java 8: CITA Es un lenguaje de programación informática de propósito general que es concurrente, basado en clases y orientado a objetos.
 \item Postgresql: [17] Es un sistema de base de datos objeto-relacional que tiene las características de los sistemas de base de datos propietarios tradicionales con mejoras de los sistemas de base de datos de la nueva generación.
 \item ActiveJDBC: CITARápido ORM (Object Relational Mapping) para desarrollo ágil. Es una implementación en el lenguaje Java del patrón arquitectural Active Record.
 \item JUnit: CITA Es un framework para escribir tests (pruebas de software). Es una implementación de la arquitectura xUnix para unit testing.
 \item Spark:CITA Es un framework inspirado en Sinatra para crear aplicaciones web con Java 8.
 \item Sinatra:CITA es un framework para aplicaciones web de software libre y código abierto con una DSL (Domain Specific Language) escrito en Ruby on Rail.
 \item PivotalTracker: CITA Es una interesante herramienta para la gestión de proyectos ágiles.
%  \item SCRUM [23]: una metodología ágil para desarrollar proyectos de software.
 \item Maven: herramienta de software para la gestión y construcción de proyectos Java creada por Jason van Zyl, de Sonatype, en 2002. Es similar en funcionalidad a Apache Ant (y en menor medida a PEAR de PHP y CPAN de Perl), pero tiene un modelo de configuración de construcción más simple, basado en un formato XML.
 \item GanttProject: CITA herramienta gratuita para crear una completa planificación de un proyecto de forma visual. Todo queda bajo control en GanttProject, desde los recursos necesarios en forma de personal, los días festivos, hasta dividir el proyecto en un árbol de tareas y asignar a cada uno los recursos oportunos.
\end{itemize}



\section{Evaluación de la propuesta y resultados}\label{resultado}
Para evaluar la propuesta, se tuvieron en cuenta los resultados obtenidos en cada uno de los proyectos desarrollados por los  
estudiantes, los aportes de los docentes involucrados, y también, se generó una encuesta a fin de reflejar el trabajo individual y grupal, 
la adopción de las nuevas tecnologías y
la aplicación de Scrum. La encuesta fue compleatada por 45 estudiantes que cursaron las asignaturas de Ingeniería de Software durante
los años 2014 y 2015. En la tabla 1 se presenta la encuesta brindada a los estudiantes.
Luego de realizar y analizar las encuestas, y evaluando el desempeño de los grupos es posible destacar lo siguiente: Todos los
grupos se mostraron muy entusiasmados e interesados con la propuesta. Algunos grupos lograron ampliamente
los objetivos iniciales (85\%) y a otros les resultó más costoso lograr adaptarse a las nuevas herramientas.
La modalidad del taller, exige tiempo de investigación fuera del horario de clases. Un par de grupos tuvieron un desempeño sobresaliente, 
obligando al cuerpo docente a proponer nuevas funcionaliades y nuevas herramientas.
La mayoria de los estudiantes desconocí­an herramientas como Git, JUnit, Pivotal Tracker, ActiveJDBC. Si bien muchas de ellas son citadas en las
asignaturas de Ingenierí­a de Software,
fue dentro de este proyecto el primer contacto con las mismas. Un gran porcentaje de los grupos expresa que se siente capaz de escalar esta experiencia a un proyecto de dimensiones mayores, aplicando las mismas tecnologias.
El trabajo en grupo fue altamente positivo. Se pudo observar el desempeño de cada uno de los integrantes de cada grupo ayudado por una de las 
herramientas utilizadas (Git). Si bien, desde el cuerpo docente se realizaba un seguimiento particular de cada 
estudiante, se le otorgó  al grupo autonomí­a para trabajar, de acuerdo a lo establecido por Scrum.
En la mayoría de los casos, ellos mismos se dividieron las tareas, por ejemplo cada uno se dedicaba a investigar una herramienta particular 
para 
luego mostrarla al resto del grupo. Git fue una de las herramientas
más resistidas en un comienzo. Esto, en parte se debe a que Git  establece una nueva modalidad de trabajo, en particular para trabajos 
distribuidos. En un comienzo provocaba problemas en muchos grupos, hasta que pudieron adoptarla, y finalmente sacar a flote los beneficios  
del versionado establecido por Git. 
Con respecto a la metodología de desarrollo utilizada Scrum, todos los grupos pudieron seguirla sin inconvenientes, comprendieron los roles dentro 
del proceso, sus obligaciones como parte de un equipo de desarrollo y lograron producir el software acordado.
\end{multicols}
\fbox{
\begin{minipage}[b][1\height]%
[t]{1\textwidth} 


\textbf{Encuesta de finalización de cuatrimetre 3er. año} \\ 
1. ¿Cúales fueron las materias que cursaste en este cuatrimestre ?\\
2. ¿Cuáles regularizaste?\\
3. ¿Te pareció interesante la propuesta del taller?\\
      Si   -  No - Parcilmente . Justifique en cada caso.\\
4. ¿Cúantas horas semanales promedio le dedicaste al taller?\\
5. ¿Se comprendió  el objetivo del taller al momento de presentarlo? 
      Si   -  No - Parcilmente \\
6. ¿Habías realizado algún proyecto de materia utilizando alguna metodología de desarrollo? Si(Cuál) - No\\
7. ¿ Todos los integrantes del grupo trabajaron de igual manera? Si – No – En ocasiones\\
8. Que porcentaje del total aportaste al proyecto.\\
9. En general, como calificas la experiencia de trabajo grupal. Excelente – Buena – Regular o Mala\\
10. Del 10 al 1, cuanto conocías de las herramientas antes de comenzar con el taller:\\
    a - Lenguaje de programación Java\\
    b - IDEs tales como Eclipse – Netbeans\\
    c - Maven\\
    d - Spark web framework\\
    f - Git\\
    g - ActiveJDBC\\
11. Describe muy brevemente parqué utilizaste cada una de las herramientas anteriores.\\
12. Del 10 al 1 califica la utilidad de cada una de las herramientas\\ utilizadas respecto a tu experiencia en este cuatrimestre.\\
13 ¿Cuáles fueron las herramientas que te resultaron más\\ fáciles de adoptar y cuáles fueron las más difíciles.?\\
14. Con respecto a las otras asignaturas de 3ero. ¿Pudiste relacionarlas con el taller ?\\
    a - Base de datos  Si – No – Parcilmente\\
    b - Diseño de Algoritmos SI – No – Parcialmente\\
15. ¿Tuvieron problemas con el tiempo y las entregas parciales?\\
16. Te sentis capaz de encarar un proyecto similar a mayor escala.\\

\end{minipage}}

 

   

\begin{multicols}{2} 

\section{Conclusiones} \label{conclusion}
La propuesta realizada en este trabajo, se fundamenta en el desarrollo de un proyecto integrador de ingeniería de software, sobre un dominio 
particular, con estudiantes de tercer año de las carreras de computación. Fundamentalmente, la propuesta está centrada en integrar todos los 
conocimientos adquiridos por el estudiante, y con el fuerte propósito de que se analicen y seleccionen un conjunto de herramientas altamente 
utilizadas en la actualidad en la industria de software.
Con claridad, puede verse que los resultados obtenidos son muy positivos. Esta propuesta permite una capacitación más completa e integral del 
estudiante, en virtud de que este, pueda integrar fuertemente los conceptos teóricos con los prácticos, visualizar el comportamiento y 
reacción del mercado ante las nuevas tecnologías, y reflexionar su impacto. 
Los objetivos planteados en la propuesta se consiguen en su totalidad, y en muchos casos, superan las expectativas que definió el equipo 
docente. Ello en el sentido, de que algunos estudiantes presentan tanto interés y dedicación, que alcanzan a superar los objetivos y proponen 
nuevos desafíos que sirven tanto para mejorar sus proyectos, como para una superación de este trabajo. 
Aplicando esta propuesta, se consigue que el estudiante se convierta en un actor más activo de su propio proceso de enseñanza y aprendizaje, 
a partir de la investigación a la que se somete para determinar las ventajas y desventajas de las herramientas y técnicas seleccionadas para 
el desarrollo de su proyecto. Luego, el estudiante es capaz de analizar y seleccionar las herramientas que más se adecúen al proceso de 
automatización bajo su desarrollo. 
También, es muy importante destacar que la simulación del desarrollo de un proyecto de software real, con un ambiente de trabajo en equipo y 
colaborativo, favorece a que los estudiantes se auto-gestionen, distribuyan sus tareas y tiempos, analicen riesgos, y que puedan ser capaces de 
aplicar acciones correctivas para controlar las desviaciones en la planificación original. Siendo los docentes, los encargados de realizar un 
seguimiento muy cercano de cada grupo, controlando permanentemente que se adecúen a la planificación prevista y evitando que el proyecto  
adquiera complejidad en demasia
o que crezca en tamaño de forma descontrolada, ya que no es objetivo de este trabajo aumentar su carga horaria  en perjuicio de su 
rendimiento académico general en la carrera.
En este año, estamos trabajando en recabar información de los estudiantes, muchos de ellos actualmente egresados, que en 2014 y 2015 desarrollaron el proyecto integrador de las asignaturas de ingeniería de software bajo esta propuesta, y que actualmente trabajan en la industria del software, con el fin de que nos puedan contar sus experiencias y realizar mediciones de cuanto los favoreció la aplicación de 
esta metodología. También pretendemos contactarnos con empresas de desarrollo empleadoras donde trabajen nuestros recientes graduados, y 
obtener su opinión, expectativas y sugerencias para mejorar la formación de los actuales estudiantes.


\begin{thebibliography}{10}

\bibitem{piimeg}
Proyectos de Innovación e Investigación para el Mejoramiento de la Enseñanza de Grado (PIIMEG), Secretarías de Ciencia y Técnica, y Académica, UNRC.:
\begin{itemize}
 \item M. Daniele. D. Romero. Definición y uso de Plantillas Genéricas para la descripción de Casos de Uso. RR Nº 302/04. 2004. 
 \item M. Daniele, D. Romero. Evolución de Plantillas Genéricas para la descripción de Casos de Uso a Plantillas Genéricas para Análisis y Diseño. RR Nº 109/05. 2005. 
 \item M. Daniele. D. Romero. La enseñanza de gestión de proyectos de software y la aplicación de herramientas que favorezcan su automatización. RR Nº 499/06. (01/08/2006, 31/07/2008).
 \item M. Daniele. F. Zorzan. Estimación y Planificación de Proyectos de Software versus duración de proyectos finales en la carrera Analista en Computación. RR Nº 171/11. (2011, 2012).
 \item M. Daniele. F. Zorzan. Causas que producen que los estudiantes de Computación retrasen la culminación de su Trabajo Final. RR Nº 923/12. (2013,2014).
\end{itemize}

\bibitem{chino}
Fabio Zorzan, Mariana Frutos, Ariel Arsaute, Marcela Daniele, Paola Martellotto, Marcelo Uva, Carlos Luna “Delayed Completion of Final Project of the Career Computer Analyst: Seeking its Causes”. XX Congreso Iberoamericano de Educación Superior (CIESC 2012), en el Marco de la XXXVIII Conferencia Latinoamericana en Informática – CLEI 2012 - Octubre 1 al 5 de 2012 - Medellín, Colombia. ISBN 978-1-4673-0792-5.

\bibitem{pressman}
Roger S Pressman. Libro: Software Engineering: A Practitioner's Approach. 8th Edition. McGraw-Hill Education. 2014.

\bibitem{jalote}
An Integrated Approach to Software Engineering. Pankaj Jalote. Springer 2006. 

\bibitem{yourdon}
E. Yourdon. Libro: Análisis estructurado moderno. Prentice-Hall Hispanoamericana, 1993.

\bibitem{meyer} Meyer Bertrand. Object Oriented Software Construction. Prentice Hall. 1997.

\bibitem{pu}
Ivar Jacobson, Grady Booch, James Rumbaugh. Libro: El proceso unificado de desarrollo de software. The Addison-Wesley, 2000

\bibitem{agile}
Scrum in Action: Agile Software Project Management and Development. Andrew Pham, Phuong Van Pham. Course Technology Ptr. 2011.

\bibitem{metagiles} Highsmith Jim. Agile Software Development Ecosystems. Addison-Wesley 2002. ISBN:0201760136
\bibitem{uml} Grady Booch. Libro: The Unified Modeling Language User Guide. The Addison-Wesley, 2005.


\bibitem{github} https://github.com
\bibitem{git} https://git-scm.com
\bibitem{java8} http://www.oracle.com
\bibitem{postgresql} http://www.postgresql.org.es
\bibitem{activejdbc} http://javalite.io/activejdbc
\bibitem{junit} http://junit.org
\bibitem{spark} http://sparkjava.com
\bibitem{sinatra} http://www.sinatrarb.com
\bibitem{pivotaltracker} https://www.pivotaltracker.com
\bibitem{maven}https://maven.apache.org
\bibitem{gantproject} https://gantt-project-management.com



\end{thebibliography}
 
 
\end{multicols}


\end{document}
