% \subsection{Organización y planificación}
% El poyecto integrador es planificado para ejecutarse durante el primer cuatrimestre del año. Cabe señalar
% que dentro del equipo docnete de la asignatura, se ha seleccionado a un docente  con dedicación part-time para la coordinación general 
% del poyecto integrador. La selección de este docente fue motivada por la vinculación que  el mismo posee con la industria de desarrollo de software. 
% Este docente es fundador de una empresa de desarrollo de software local y está habituado a formar parte de equipos de trabajo
% con desarrolladores a nivel internacional en proyectos bajo la modalidad outsorcing y freelance.
% 
% Uno de los objetivos del proyecto integrador es la aplicación de una metodología ágil, es por ello que el proyecto está guiado por Srum.
% En cada una de las fases del ciclo de vida, se aplica una herramienta específica. No es objetivo de este 
% proyecto complejizar ni sobrecargar tiempos de desarrollo. Sino que se focaliza en que cada 
% grupo de estudiantes pueda completar el trabajo, usando las tecnologías propuestas.  
Se formaron equipos de tres estudiantes, más el Scrum Master\cite{agile}, rol ocupado por un docente de la asignatura.\\
El proyecto desarrollado en el año 2014 consistió en la construcción de un  sistema web en donde un usuario  pueda loguearse y publicar avisos de compra
 y venta de vehículos. En el año 2015, el proyecto desarrollado fue la construcción de una vesión del juego \textit{Cuatro en línea}. 
 El diseño del problema se planteó de una manera simplificada evitando el crecimiento del proyecto de manera descontrolada.
 El equipo docente estableció un Product Backlog inicial, y en la medida de que cada grupo avanzaba se incluían 
 nuevas funcionalidades. Todos los grupos asistían a un encuentro semanal de dos horas donde se introducían las herramientas y se establecían los 
 objetivos según la planificación correspondiente. Durante uno de los encuentros, los ayudantes alumno, con los que cuentan las asignaturas
 involucradas en este trabajo, elaboraron un taller sobre una de las herramientas seleccionadas. Esto permitió realizar aportes a su formación docente, 
 y al mismo tiempo brindó  una mirada diferente al aprovechamiento de estas tecnologías.
 En el resto de los encuentros, los docentes realizaron el seguimiento de cada grupo de estudiantes.
 A continuación se presenta, a modo de ejemplo, la planificación para el proyecto integrador
llevado a cabo durante el año 2014.

\begin{itemize}
 \item Marzo 18 - Presentación general del taller e introducción de Git.
\item Abril 1 - SCRUM y Definnición del SRS (Sofware Requeriment Specification). PivotalTracker - Creación del Product Backlog.
\item Abril 8 - Introducción a Java - Eclipse / ActiveJDBC / Postgresql.
\item Abril 15 - Taller de  Maven y JUnit.
\item Abril 22 - Diseño / Implementación.
\item Abril 29 - Diseño / Implementación.
\item Mayo 6 - Taller de JUnit - definición de la Test Suite.
\item Mayo 13 - Diseño / Implementación.
\item Mayo 20 - Checkpoint (Test Suite running).
\item Mayo 27 - Taller Spark - Sinatra e incorporación de la capa web.
\item Junio 3 - Implementación 
\item Junio 10 - Implementación
\item Junio 17 - Presentación de grupos
\end{itemize}

\subsection{Tecnologías y herramientas utilizadas}
La industria del software se encuentra en continua expansión. Esto requiere la incorporación de nuevas tecnologías para generar
nuevas tecnologías. Sólo incorporando nuevas herramientas de soporte para estos procesos es que se logra producir sistemas para millones
de usuarios, como por ejemplo las redes sociales. \\
Las herramientas utilizadas en este proyecto fueron las siguientes:
 \begin{itemize}
 \item Github\cite{github}: plataforma de desarrollo colaborativo de software para alojar proyectos utilizando el sistema de control de versiones Git. El código se almacena de forma pública, aunque también se puede hacer de forma privada, creando una cuenta de pago. 
 \item Git\cite{git}: sistema de control de versiones distribuido, libre y gratuito.
 \item Java 8\cite{java8}: lenguaje de programación informática de propósito general que es concurrente, basado en clases y orientado a objetos.
 \item Postgresql\cite{postgresql}: sistema de base de datos objeto-relacional con características de los sistemas de base de datos propietarios tradicionales con mejoras de los sistemas de base de datos de la nueva generación.
 \item ActiveJDBC\cite{activejdbc}: ORM (Object Relational Mapping) para desarrollo ágil. Es una implementación en el lenguaje Java del patrón arquitectural Active Record.
 \item JUnit\cite{junit}: framework para escribir tests (pruebas de software). Es una implementación de la arquitectura xUnix para unit testing.
 \item Spark\cite{spark}: framework inspirado en Sinatra para crear aplicaciones web con Java 8.
 \item Sinatra\cite{sinatra}: framework para aplicaciones web de software libre y código abierto con una DSL (Domain Specific Language) escrito en Ruby on Rail.
 \item PivotalTracker\cite{pivotaltracker}: herramienta para la gestión de proyectos ágiles.
%  \item SCRUM [23]: metodología ágil para desarrollar proyectos de software.
 \item Maven\cite{maven}: herramienta de software para la gestión y construcción de proyectos Java creada por Jason van Zyl, de Sonatype, en 2002. Es similar en funcionalidad a Apache Ant (y en menor medida a PEAR de PHP y CPAN de Perl), con un modelo de configuración de construcción más simple, basado en un formato XML.
 \item GanttProject\cite{gantproject}: herramienta para crear una completa planificación de un proyecto de forma visual. Todo queda bajo control en GanttProject, desde los recursos necesarios en forma de personal, los días festivos, hasta dividir el proyecto en un árbol de tareas y asignar a cada uno los recursos oportunos.
\end{itemize}
