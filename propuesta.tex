\section{Desripción de la propuesta} \label{propuesta}

En este trabajo se presenta una experiencia desarrollada durante los años 2014 y 2015 dentro de las asignaturas Análisis y Diseño de
Sistemas e 
Ingeniería de Software perteneciente a las carreras de computación dictadas dentro de la Universidad Nacional de Río Cuarto.
La propuesta consiste 
en el desarrollo de un proyecto-taller en donde, además de integrar los contenidos teórico-prácticos 
estudiados, se incorporen un conjunto de tecnologías y herramientas utilizadas en la industria del desarrollo de software actual,
permitiendo
establecer un mejor
puente entre el futuro egresado y las empresas de desarrollo de software. \\


\subsection{Objetivos de la propuesta}

 \begin{itemize}
 \item Integrar en un  proyecto de desarrollo de software los contenidos estudiados en las asignaturas pertenecientes al tercer 
   año de las carreras de Analista, Profesorado y  Licenciatura en Ciencias de la Computación. La planificación CITA del proyecto esta 
   fuertemente articulada con la planificación de la asignatura.
 \item Aplicar una metodología de desarrollo ágil como Scrum.
 \item Simular un ambiente de trabajo con características similares a las de un ambiente de trabajo real. Para ello se asignan responsabilidades
 específicas a cada uno de los integerantes de cada grupo. Si bien los objetivos particulares para cada grupo deben realiarse en equipo, 
 se realiza un seguimiento particular de cada uno de los alumnos. Su integración con el resto, su compromiso y forma de trabajo.
 \item Desarrollar en los alumnos habilidades para poder adoptar y sacar provecho de un set de herramientas utilizadas en la industria del software. 
 
 \end{itemize}
 
\subsection{Organización y planificación}
El taller se organiza para ser llevado adelante durante el primer cuatrimestre del año. Cabe señalar
que dentro del grupo de docente de la asignatura, se ha seleccionado a un docente  con dedicación part-time para la coordinación general 
del taller. La selección de este docente fue motivada por la vinculación que posee el mismo con la industria de desarrollo. 
 Este docente es fundador de una empresa de desarrollo de software local y está habituado a formar parte de grupos de trabajo
con desarrolladores a nivel internacional. // 
Uno de los objetivos del taller es la aplicación de una metodología ágil. Es por ello que el proyecto estará guíado por Srum.
En cada una de las fases del ciclo de vida, se aplicará una herramienta específica detallada en la sección CITA. No es objetivo de este 
proyecto el que 
los alumnos realicen un trabajo que se exceda  en cuanto a complejidad y tiempo que les puediere insumir. El mismo se focaliza en que cada 
grupo pueda completar el trabajo, usando las tecnologías propuestas. A continuación se presenta la planificación para el proyecto-integrador
durante el año 2015.




\subsection{Tecnologías y herramientas utilizadas}
La industria del software se encuentra en continua expansión. Esto requiere la incorporación de nuevas tecnologías para generar
nuevas tecnologías. Sólo incorporando nuevas herramientas de soporte para estos procesos es que se logra producir sistemas para millones
de usuarios como por ejemplo las redes sociales Facebook o Twitter. \\
Las herramientas utilizadas en este proyecto fueron las siguientes:
 \begin{itemize}
 \item Github:CITA Es una plataforma de desarrollo colaborativo de software para alojar proyectos utilizando el sistema de control de versiones Git [14]. El código se almacena de forma pública, aunque también se puede hacer de forma privada, creando una cuenta de pago. 
 \item Git:CITA Es un sistema de control de versiones distribuido, libre y gratuito.
 \item Java 8: CITA Es un lenguaje de programación informática de propósito general que es concurrente, basado en clases y orientado a objetos.
 \item Postgresql: [17] Es un sistema de base de datos objeto-relacional que tiene las características de los sistemas de base de datos propietarios tradicionales con mejoras de los sistemas de base de datos de la nueva generación.
 \item ActiveJDBC: CITARápido ORM (Object Relational Mapping) para desarrollo ágil. Es una implementación en el lenguaje Java del patrón arquitectural Active Record.
 \item JUnit: CITA Es un framework para escribir tests (pruebas de software). Es una implementación de la arquitectura xUnix para unit testing.
 \item Spark:CITA Es un framework inspirado en Sinatra para crear aplicaciones web con Java 8.
 \item Sinatra:CITA es un framework para aplicaciones web de software libre y código abierto con una DSL (Domain Specific Language) escrito en Ruby on Rail.
 \item PivotalTracker: CITA Es una interesante herramienta para la gestión de proyectos ágiles.
%  \item SCRUM [23]: una metodología ágil para desarrollar proyectos de software.
 \item Maven: herramienta de software para la gestión y construcción de proyectos Java creada por Jason van Zyl, de Sonatype, en 2002. Es similar en funcionalidad a Apache Ant (y en menor medida a PEAR de PHP y CPAN de Perl), pero tiene un modelo de configuración de construcción más simple, basado en un formato XML.
 \item GanttProject: CITA herramienta gratuita para crear una completa planificación de un proyecto de forma visual. Todo queda bajo control en GanttProject, desde los recursos necesarios en forma de personal, los días festivos, hasta dividir el proyecto en un árbol de tareas y asignar a cada uno los recursos oportunos.
\end{itemize}


