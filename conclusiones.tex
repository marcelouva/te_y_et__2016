\section{Conclusiones} \label{conclusion}
La propuesta realizada en este trabajo, se fundamenta en el desarrollo de un proyecto integrador de ingeniería de software, sobre un dominio 
particular, con estudiantes de tercer año de las carreras de computación. Fundamentalmente, la propuesta está centrada en integrar todos los 
conocimientos adquiridos por el estudiante, y con el fuerte propósito de que se analicen y seleccionen un conjunto de herramientas altamente 
utilizadas en la actualidad en la industria de software.
Con claridad, puede verse que los resultados obtenidos son muy positivos. Esta propuesta permite una capacitación más completa e integral del 
estudiante, en virtud de que este, pueda integrar fuertemente los conceptos teóricos con los prácticos, visualizar el comportamiento y 
reacción del mercado ante las nuevas tecnologías, y reflexionar su impacto. 
Los objetivos planteados en la propuesta se consiguen en su totalidad, y en muchos casos, superan las expectativas que definió el equipo 
docente. Ello en el sentido, de que algunos estudiantes presentan tanto interés y dedicación, que alcanzan a superar los objetivos y proponen 
nuevos desafíos que sirven tanto para mejorar sus proyectos, como para una superación de este trabajo. 
Aplicando esta propuesta, se consigue que el estudiante se convierta en un actor más activo de su propio proceso de enseñanza y aprendizaje, 
a partir de la investigación a la que se somete para determinar las ventajas y desventajas de las herramientas y técnicas seleccionadas para 
el desarrollo de su proyecto. Luego, el estudiante es capaz de analizar y seleccionar las herramientas que más se adecúen al proceso de 
automatización bajo su desarrollo. 
También, es muy importante destacar que la simulación del desarrollo de un proyecto de software real, con un ambiente de trabajo en equipo y 
colaborativo, favorece a que los estudiantes se auto-gestionen, distribuyan sus tareas y tiempos, analicen riesgos, y que puedan ser capaces de 
aplicar acciones correctivas para controlar las desviaciones en la planificación original. Siendo los docentes, los encargados de realizar un 
seguimiento muy cercano de cada grupo, controlando permanentemente que se adecúen a la planificación prevista y evitando que el proyecto  
adquiera complejidad en demasia
o que crezca en tamaño de forma descontrolada, ya que no es objetivo de este trabajo aumentar su carga horaria  en perjuicio de su 
rendimiento académico general en la carrera.
En este año, estamos trabajando en recabar información de los estudiantes, muchos de ellos actualmente egresados, que en 2014 y 2015 desarrollaron el proyecto integrador de las asignaturas de ingeniería de software bajo esta propuesta, y que actualmente trabajan en la industria del software, con el fin de que nos puedan contar sus experiencias y realizar mediciones de cuanto los favoreció la aplicación de 
esta metodología. También pretendemos contactarnos con empresas de desarrollo empleadoras donde trabajen nuestros recientes graduados, y 
obtener su opinión, expectativas y sugerencias para mejorar la formación de los actuales estudiantes.
