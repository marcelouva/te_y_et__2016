\section{Introducción}

Durante la última década, la relación Universidad-Industria de desarrollo de software se ha vuelto cada vez más estrecha. Pequeñas, medianas y grandes empresas se acercan a menudo a las universidades en busca de analistas, programadores, ingenieros de software, etc.
El constante crecimiento y expansión del mercado informático, en todos los ámbitos, hace cada vez más visible la necesidad de contar con 
recursos humanos calificados. Paralelamente, dicho crecimiento ha propiciado en la industria, la generación y adopción de nuevas 
metodologías de desarrollo y nuevas modalidades de trabajo por ej. desarrollos outsourcing.
Junto con éstas, han surgido nuevas tecnologías brindando el soporte necesario para las actividades de administración, 
gestión, planificación, implementación, diseño, automatización del testing, seguimiento y control de las tareas que hacen a la producción 
del software, entre otras. Muchas de éstas tecnologías poco a poco han ido  incorporandose a las currículas de las carreras de informática. De esta manera se ha establecido una relación de necesidad entre Universidad e Indrustria de desarrollo de sofware en ambos sentidos.
 
 
En el marco de los proyectos PIIMEG CITA (Proyectos de Investigación e Innovación para el Mejoramiento de la Enseñanza de Grado) pertenecientes a la Universidad
 Nacional de Río Cuarto se han realizado una serie de propuestas en pos de detectar, analizar y ejecutar acciones concretas con el 
 fin de realizar aportes para solucionar problemáticas observadas en las asignaturas de 3er. año de las carreras de Analista, Profesorado y 
 Licenciatura en Ciencias de la Computación. Algunas de las acciones realizadas se resumen a continuación:
 \begin{itemize}
 \item Desarrollo e un Proyecto-taller de integración de todas las asignaturas de 3er. año (Análisis y Diseño de Sistemas, Base de Datos y 
  Diseño de algoritmos. El problema observado en este caso radicaba en que los alumnos debían realizar un proyecto final  para cada una de las asignaturas
 de 3er. año. En algunos casos, la carga horaria de la asignatura tenía contemplado esto y en otras no. Esto producía que  alumnos
quedaran libres en algunas materias por no poder cumplir con los requisitos de las mismas. O bien, ellos mismos optaran por hacer unas u otras, debido a la exigencia horaria requerida.  
Para evitar todo lo anterior, se diseño un proyecto-taller integrador único para todas las asignaturas de 3er. año. Cada asignatura incluiría en éste 
los contenidos a evaluar que requiriera. El proyecto-taller consistiría en el desarrollo de un
 sistema informático en donde la asignatura Base de Datos se encargaría de diseñar y generar la base de datos.
 La asignatura Diseño de Algoritmos 
 debía establecer 
 algún requerimiento que implicara la necesidad de la utilización de técnicas algorítmicas estudiadas durante la cursada y, finalmente  Análisis y Diseño de Sistemas
 se encargaría del seguimiento y control del proceso de desarrollo. El dominio del proyecto sería definido por los cuerpos docentes de las tres asignaturas.CITA
 \item Definición de una plantilla para guiar la documentación en lo que respecta a proyectos desarrollados con metodologías ágiles.CITA 
\item Estudio y análisis de causas de retraso de la finalización de los Trabajos Finales CITA. 
\item Propuesta para realizar el Trabajo Final en los tiempos planificados CITA.
\end{itemize}

En este trabajo se presenta una experiencia desarrollada durante los años 2014 y 2015 dentro de las asignaturas Análisis y Diseño de Sistemas e Ingeniería de Software. 
La propuesta consiste en el desarrollo de un proyecto-taller en donde, además de integrar los contenidos teórico-prácticos abordados en las 
asignaturas de tercer año, se incorporen un conjunto de tecnologías utilizadas en la industria del desarrollo de software actual. \\



El resto del trabajo está organizado de la siguiente manera: en la sección 2 se presentan los fundamentos de la propuesta y se plantean los 
objetivos. La sección 3 presenta la metodología de desarrollo aplicada. En la sección 4 se presenta la propuesta, su planificación y 
las tecnologías y herramientas utlizadas. En la sección 5 se realiza una evalución de la propuesta, y finalmente, las conclusiones.