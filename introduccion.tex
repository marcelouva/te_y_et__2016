\section{Introducción}
Durante la última década, la relación de la Universidad con la  industria del desarrollo de software se ha vuelto cada vez más estrecha. Pequeñas, medianas y grandes empresas se acercan continuamente a las universidades en busca de analistas, programadores, ingenieros de software, etc.
El constante crecimiento y expansión del mercado informático, en todos los ámbitos, hace cada vez más visible la necesidad de contar con 
recursos humanos calificados. Paralelamente, dicho crecimiento ha propiciado en la industria, la generación y adopción de nuevas 
metodologías de desarrollo y conjuntamente, nuevas modalidades de trabajo como desarrollos outsourcing y freelance.
Nuevas tecnologías han surgido con el objeto de dar el soporte necesario para las actividades de administración, 
gestión, planificación, implementación, diseño, automatización de la prueba, seguimiento y control de las tareas que hacen a la producción 
del software. Muchas de estas tecnologías poco a poco han ido  incorporándose en las currículas de las carreras de informática. 
De esta manera, se ha establecido una relación de necesidad entre Universidad e indrustria de desarrollo de sofware en ambos sentidos.
En el marco de los proyectos Proyectos de Investigación e Innovación para el Mejoramiento de la Enseñanza de Grado (PIIMEG)\cite{piimeg} 
pertenecientes a la Universidad Nacional de Río Cuarto (UNRC), se han llevado adelante una serie de propuestas en pos de detectar, analizar y ejecutar 
acciones concretas con el fin de realizar aportes para solucionar problemáticas observadas en las asignaturas de tercer año de las carreras de 
Analista, Profesorado y Licenciatura en Ciencias de la Computación dictadas en la UNRC. 
En este trabajo se presenta una experiencia desarrollada durante los años 2014 y 2015 en el marco de las asignaturas de Ingeniería de Software. 
La propuesta consiste en el desarrollo de un proyecto donde, además de integrar los contenidos teórico-prácticos abordados en las 
asignaturas de tercer año, se incorporen un conjunto de tecnologías utilizadas en la industria del desarrollo de software actual. 
El resto del trabajo está organizado de la siguiente manera: en la sección \ref{fundamenta} se presentan los 
fundamentos de la propuesta. La sección \ref{metodologia} presenta la metodología de desarrollo aplicada. 
En la sección \ref{propuesta} se presenta la propuesta, planificación, 
tecnologías y herramientas utlizadas. En la sección \ref{resultado} se realiza una evalución de la propuesta, 
y finalmente, las conclusiones.