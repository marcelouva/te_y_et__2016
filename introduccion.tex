\section{Introducción}
Durante la última década, la relación Universidad e Industria de desarrollo de software se ha vuelto cada vez más estrecha. Pequeñas, medianas y grandes empresas se acercan continuamente a las universidades en busca de analistas, programadores, ingenieros de software, etc.
El constante crecimiento y expansión del mercado informático, en todos los ámbitos, hace cada vez más visible la necesidad de contar con 
recursos humanos calificados. Paralelamente, dicho crecimiento ha propiciado en la industria, la generación y adopción de nuevas 
metodologías de desarrollo y conjuntamente, nuevas modalidades de trabajo por ej. desarrollos outsourcing.
Nuevas tecnologías han surgido con el objeto de dar el soporte necesario para las actividades de administración, 
gestión, planificación, implementación, diseño, automatización del testing, seguimiento y control de las tareas que hacen a la producción 
del software. Muchas de estas tecnologías poco a poco han ido  incorporándose en las currículas de las carreras de informática. De esta manera se ha establecido una relación de necesidad entre Universidad e Indrustria de desarrollo de sofware en ambos sentidos.
En el marco de los proyectos PIIMEG CITA (Proyectos de Investigación e Innovación para el Mejoramiento de la Enseñanza de Grado) pertenecientes a la Universidad Nacional de Río Cuarto se han llevado adelante una serie de propuestas en pos de detectar, analizar y ejecutar acciones concretas con el fin de realizar aportes para solucionar problemáticas observadas en las asignaturas de 3er. año de las carreras de Analista, Profesorado y Licenciatura en Ciencias de la Computación. 
En este trabajo se presenta una experiencia desarrollada durante los años 2014 y 2015 dentro de las asignaturas Análisis y Diseño de Sistemas. 
La propuesta consiste en el desarrollo de un proyecto-taller en donde, además de integrar los contenidos teórico-prácticos abordados en las 
asignaturas de tercer año, se incorporen un conjunto de tecnologías utilizadas en la industria del desarrollo de software actual. 
El resto del trabajo está organizado de la siguiente manera: en la sección 2 se presentan los fundamentos de la propuesta y se plantean los 
objetivos. La sección 3 presenta la metodología de desarrollo aplicada. En la sección 4 se presenta la propuesta, planificación, 
tecnologías y herramientas utlizadas. En la sección 5 se realiza una evalución de la propuesta, y finalmente, las conclusiones.