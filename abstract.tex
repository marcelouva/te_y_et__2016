
\begin{abstract}
Durante la última década, la relación Universidad-Industria de desarrollo de software se ha vuelto cada vez más estrecha. Pequeñas, medianas y grandes empresas se acercan a menudo a las universidades en busca de analistas, programadores, ingenieros de software, etc.
El crecimiento y expansión del mercado informático requiere recursos humanos calificados. Paralelamente, dicho crecimiento ha propiciado en la industria, la generación y adopción de nuevas 
metodologías de desarrollo, nuevas modalidades de trabajo por ejemplo  desarrollos outsourcing.
Junto con éstas, han surgido nuevas tecnologías brindando el soporte necesario para actividades de administración, 
gestión, planificación, implementación, diseño, testing  y control en  la producción 
del software, entre otras. 
En este trabajo se presenta una experiencia desarrollada durante los años 2014 y 2015 dentro de las asignaturas Análisis 
y Diseño de Sistemas e Ingeniería de Software. La propuesta consiste en el desarrollo de un proyecto-taller integrador que aplica un 
set de herramientas y tecnologías utilizadas en la industria del desarrollo de software actual, estableciendo un puente más 
directo entre formación académica y necesidad del mercado.

\end{abstract}
