
\begin{abstract}
Durante la última década, la relación Universidad e industria de desarrollo de software se ha vuelto cada vez más estrecha. 
Pequeñas, medianas y grandes empresas se acercan continuamente a las universidades en busca de recursos humanos calificados.
El crecimiento y expansión del mercado informático ha propiciado la generación y adopción de nuevas 
metodologías de desarrollo, nuevas modalidades de trabajo como desarrollos outsourcing y freelance.
Al mismo tiempo, han surgido nuevas tecnologías que brindan el soporte necesario para las actividades de administración, 
gestión, planificación, implementación, diseño, prueba  y control en  la producción 
del software. 
En este trabajo se presenta una experiencia desarrollada durante los años 2014 y 2015 en el marco de las asignaturas de Ingeniería
de Sofware. La propuesta consiste en el desarrollo de un proyecto integrador que aplica un 
conjunto de herramientas y tecnologías de gran impacto en la industria del desarrollo de software actual, estableciendo un puente directo  
entre formación académica y necesidad del mercado.\\ \\
\textbf{Palabras Claves}: Ingeniería de Software, Análisis y Diseño de Sistemas, Industria del Software y  Scrum. 

\end{abstract}
