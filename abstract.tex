
\begin{abstract}
Durante la última década, la relación Universidad e Industria de desarrollo de software se ha vuelto cada vez más estrecha. Pequeñas, medianas y grandes empresas se acercan continuamente a las universidades en busca de recursos humanos calificados. El crecimiento y expansión del mercado informático ha propiciado, la generación y adopción de nuevas 
metodologías de desarrollo, nuevas modalidades de trabajo por ejemplo  desarrollos outsourcing.
Junto con éstas, han surgido nuevas tecnologías que brindan el soporte necesario para las actividades de administración, 
gestión, planificación, implementación, diseño, testing  y control en  la producción 
del software. 
En este trabajo se presenta una experiencia desarrollada durante los años 2014 y 2015 dentro de las asignaturas Análisis 
y Diseño de Sistemas. La propuesta consiste en el desarrollo de un proyecto-taller integrador que aplica un 
set de herramientas y tecnologías utilizadas en la industria del desarrollo de software actual, estableciendo un puente más 
directo entre formación académica y necesidad del mercado.

\end{abstract}
