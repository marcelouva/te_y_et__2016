\section{Evaluación de la propuesta y resultados}\label{resultado}
Para evaluar la propuesta, se tuvieron en cuenta los resultados obtenidos en cada uno de los proyectos desarrollados por los  
estudiantes, los aportes de los docentes involucrados, y también, se generó una encuesta a fin de reflejar el trabajo individual y grupal, 
la adopción de las nuevas tecnologías y
la aplicación de Scrum. La encuesta fue compleatada por 45 estudiantes que cursaron las asignaturas de Ingeniería de Software durante
los años 2014 y 2015. En la tabla 1 se presenta la encuesta brindada a los estudiantes.
Luego de realizar y analizar las encuestas, y evaluando el desempeño de los grupos es posible destacar lo siguiente: Todos los
grupos se mostraron muy entusiasmados e interesados con la propuesta. Algunos grupos lograron ampliamente
los objetivos iniciales (85\%) y a otros les resultó más costoso lograr adaptarse a las nuevas herramientas.
La modalidad del taller, exige tiempo de investigación fuera del horario de clases. Un par de grupos tuvieron un desempeño sobresaliente, 
obligando al cuerpo docente a proponer nuevas funcionaliades y nuevas herramientas.
La mayoria de los estudiantes desconocí­an herramientas como Git, JUnit, Pivotal Tracker, ActiveJDBC. Si bien muchas de ellas son citadas en las
asignaturas de Ingenierí­a de Software,
fue dentro de este proyecto el primer contacto con las mismas. Un gran porcentaje de los grupos expresa que se siente capaz de escalar esta experiencia a un proyecto de dimensiones mayores, aplicando las mismas tecnologias.
El trabajo en grupo fue altamente positivo. Se pudo observar el desempeño de cada uno de los integrantes de cada grupo ayudado por una de las 
herramientas utilizadas (Git). Si bien, desde el cuerpo docente se realizaba un seguimiento particular de cada 
estudiante, se le otorgó  al grupo autonomí­a para trabajar, de acuerdo a lo establecido por Scrum.
En la mayoría de los casos, ellos mismos se dividieron las tareas, por ejemplo cada uno se dedicaba a investigar una herramienta particular 
para 
luego mostrarla al resto del grupo. Git fue una de las herramientas
más resistidas en un comienzo. Esto, en parte se debe a que Git  establece una nueva modalidad de trabajo, en particular para trabajos 
distribuidos. En un comienzo provocaba problemas en muchos grupos, hasta que pudieron adoptarla, y finalmente sacar a flote los beneficios  
del versionado establecido por Git. 
Con respecto a la metodología de desarrollo utilizada Scrum, todos los grupos pudieron seguirla sin inconvenientes, comprendieron los roles dentro 
del proceso, sus obligaciones como parte de un equipo de desarrollo y lograron producir el software acordado.