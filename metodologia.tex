\section{Metodología de desarrollo aplicada al proyecto integrador} \label{metodologia}
% \section{Metodologías Tradicionales y Metodologías Ágiles}\label{metodologia}
En el marco de las asignaturas de Ingeniería de Sofware, se presenta una visión general sobre las diferentes metodologías utilizadas en 
el de desarrollo de 
software\cite{pressman,jalote}.
Algunos de los ciclos de vida del software estudiados son:  el lineal o secuencial, el método del Análisis Estructurado de Yourdon\cite{yourdon}, 
desarrollo basado en componentes, diseño por contratos\cite{meyer} de Bertrand Meyer, entre otros. 
Particularmente, se pone énfasis en el Proceso Unificado\cite{pu}, como método de desarrollo
orientado a objetos tradicional  y en Scrum\cite{agile}, como metodología de desarrollo ágil.\\ 
En este trabajo se presenta una modificación al proyecto de articulación de contenidos mencionado en la sección
\ref{fundamenta} meditante la 
incorporación de un 
conjunto  de herramientas 
que la industria de desarrollo de software utiliza intensamente en la actualidad. Dichas tecnologías están intimamente ligadas a la 
aplicación de 
metodologías ágiles\cite{metagiles}, especialmente con Scrum, metodología seleccionada por el equipo docente para el desarrollo de este 
proyecto integrador.
Scrum es una metodología ágil que posibilita trabajar en ambientes muy cambiantes, permitiendo replanteamientos continuos. 
Por otro lado, reduce el tiempo de producción y de comercialización del producto, aporta un gran beneficio o 
valor agregado al cliente, minimiza los riesgos de desperdiciar esfuerzo/tiempo en la construcción de artefactos que no serán utilizados 
o que no son fundamentales para el cliente. Facilita también la comunicación entre todos los integrantes del proyecto. 
La documentación producida dentro de un proyecto Srum es relativa al usuario, dueño, producto y equipo.  Scrum es un marco de 
trabajo basado sobre la premisa de que el equipo de desarrollo conocerá la mejor manera de resolver el problema que se le presenta. 
La reunión de planificación de cada conjunto de requerimientos a producir se describe en términos del resultado deseado, 
en lugar de un conjunto de criterios de ingreso, definiciones y tareas. 
Scrum se basa en una auto-organización, con un equipo multifuncional y sin líder (dentro del equipo). 
El equipo es apoyado por dos individuos quienes ocupan los roles de Scrum Master y  Product Owner. 
El Scrum Master es una especie de entrenador para el equipo, su función es ayudar a los miembros del mismo a utilizar el marco que ofrece 
la metodología para conseguir un alto nivel de productividad. Mientras que el Product Owner representa los negocios, clientes o 
usuarios. Éste, guía al equipo hacia la construcción del producto esperado.
 Los proyectos Scrum avanzan en orden a la definición de los sprints, que son las iteraciones que poseen una duración de entre dos y cuatro semanas. 
En el inicio de cada sprint, los miembros del equipo se comprometen a producir un cierto número de características que se enumeran en 
el artefacto conocido como Product Backlog del proyecto.  Al final de cada sprint, cada funcionalidad (conocida como user story) debe estar 
codificada,
 probada e integrada a una versión demo del sprint anterior. Luego se realiza una revisión, y por último se presenta la nueva funcionalidad
 frente al Product Owner y las otras partes interesadas que proporcionarán información requerida para el siguiente sprint. 
Las iteraciones han de continuar hasta obtener el producto deseado.
Como se puede observar de lo expuesto, Scrum establece una forma de trabajo en donde todo el equipo se auto-regula y en donde no hay 
una documentación establecida a priori. Cada equipo utilizará los elementos que le sean necesarios para poder llevar adelante el conjunto de 
user stories
con el cual se ha comprometido. Algunos equipos podrán utilizar diagramas UML\cite{uml}, otros utilizarán diagramas de flujos de datos, etc.
En la figura \ref{figu1} se esquematiza todo el proceso.
\section{Descripción de la propuesta} \label{propuesta}

La propuesta presentada en este trabajo consiste 
en el desarrollo de un proyecto integrador donde, además de integrar los contenidos teórico-prácticos 
estudiados en las asignaturas de Ingeniería de Sofware, se incorpore un conjunto de tecnologías y herramientas utilizadas en la industria del 
desarrollo de software actual,
permitiendo
establecer un vínculo estrecho entre el futuro egresado y las empresas de desarrollo de software. \\
\subsection{Objetivos}
A continuación se enumeran los objetivos de esta propuesta:
 \begin{itemize}
 \item Integrar en un  proyecto de desarrollo de software los contenidos trabajados en tercer año, fundamentalmente, aquellos
 pertenecientes a las asignaturas de Ingeniería de Sofware. 
 \item Aplicar una metodología de desarrollo ágil como Scrum. Aprovechando de esta manera los beneficios que conlleva la misma y a sabiendas de que 
 es una de la más utilizadas por las empresas de desarrolo de software.
 \item Simular un ambiente de trabajo con características similares a las de un ambiente de trabajo real. Para ello se establecen objetivos grupales 
 y es el mismo grupo quien se auto-gestiona. Si bien los objetivos son grupales, en el marco del proyecto integrador, 
 se realiza un seguimiento y valoración particular de cada uno de los 
 estudiantes, su integración con el resto, su compromiso, su forma de trabajo, etc.
 \item Desarrollar en los estudiantes habilidades para adoptar y obtener beneficios  de un conjunto de herramientas utilizadas en la industria del software. 
 El objetivo no es sólo que aprendan a usar una herramienta, 
%  \newpage
 sino que sean capaces de seleccionar las herramientas adecuadas en futuros proyectos.
 
 
 \subsection{Organización y planificación}
El poyecto integrador es planificado para ejecutarse durante el primer cuatrimestre del año. Cabe señalar
que dentro del equipo docente de la asignatura, se ha seleccionado a un docente  con dedicación part-time para la coordinación general 
del poyecto integrador. La selección de este docente fue motivada por la vinculación que  el mismo posee con la industria de desarrollo de software. 
Este docente es fundador de una empresa de desarrollo de software local y está habituado a formar parte de equipos de trabajo
con desarrolladores a nivel internacional en proyectos bajo la modalidad outsorcing y freelance. 
Uno de los objetivos del proyecto integrador es la aplicación de una metodología ágil, es por ello que el proyecto está guiado por Srum.
En cada una de las fases del ciclo de vida, se aplica una herramienta específica. No es objetivo de este 
proyecto agregar complejidad en demasia  ni sobrecargar tiempos de desarrollo. Sino que se focaliza en que cada 
grupo de estudiantes pueda completar el trabajo, usando las tecnologías propuestas.  
 
 \end{itemize}
% 