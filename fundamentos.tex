\section{Fundamentos}\label{fundamenta}
La planificación y ejecución de procesos de enseñanza-aprendizaje para cursos de ingeniería de software (IS), plantean un gran desafío a los docentes universitarios involucrados. La necesidad de una actualización dinámica de los contenidos no debe provocar el descuido de conceptos básicos vinculados a los principios fundamentales del desarrollo de sistemas de software.
El cuerpo docente, autor de este artículo, tiene a cargo el dictado de los cursos Análisis y Diseño de Sistemas e Ingeniería de Software, durante el tercer año de las carreras: Licenciatura y Profesorado en Ciencias de la Computación y Analista en Computación, de la Universidad Nacional de Río Cuarto.

Los conceptos básicos de IS, son introducidos en el curso de Análisis y Diseño de Sistemas. En este se presentan diferentes metodologías de desarrollo de software [ver$$$$]. 

Para el curso de Ingeniería de Software, el principal propósito es que el alumno tome conocimiento de los conceptos más avanzados de IS; desde la planificación y gestión del proyecto hasta técnicas de testing o prueba. Al mismo tiempo se cubren  conceptos transversales a las etapas de desarrollo como el gerenciamiento de la configuración de software



Esta metodología incorpora la realización de proyectos-taller, abordando específicamente el desafío de la utilización de herramientas que asistan en las actividades de gestión de proyectos de software [ver$$$$] y ayuden al mismo tiempo la compresión profunda de los temas abordados. 
El principal propósito es conseguir que los alumnos vivencien situaciones muy cercanas a la realidad y, de esta manera, disminuir la brecha entre la teoría universitaria y la realidad profesional.



El dictado de la asignatura IS se divide en clases teóricas, clases prácticas y proyectos-taller

Durante los años 2014 y 2015, se abordó importantes desafíos a cubrir en la enseñanza de ingeniería de software, introduciendo la utilización de herramientas que permitan automatizar la gestión de proyectos de software [Ver$$$$]. 

En este marco, se logró fomentar el uso de una metodología de desarrollo de software en proyectos reales. Promoviendo de esta manera la división de roles operativos y gerenciales entre los integrantes de un grupo de desarrollo de software, capacitándolos para la evaluación y selección de diferentes herramientas aplicables a las actividades de gestión de software.

En el año 2014, se presentó ……

En el año 2015, se presentó ……
