\section{Fundamentos}\label{fundamenta}
Desde hace más de una década, en el marco de los proyectos PIIMEG\cite{piimeg},
los autores de este trabajo, han realizado acciones tendientes a dar solución a diversas problemáticas identificadas en las carreras de 
computación dictadas en la UNRC. A continuación se presenta un breve resumen de las temáticas abordadas:
\begin{itemize}
 \item En el año 2004, se propuso la definición y uso de plantillas genéricas para la descripción de Casos de Uso. El objetivo fue proporcionarle al estudiante una forma clara y concreta de describir las funcionalidades de un sistema a desarrollar 
mediante soluciones genéricas a problemas similares o recurrentes. 
 \item En el año 2005, continuando en la misma  lína de investigación, se realizó 
la definición de plantillas genéricas para cubrir las etapas siguientes de un proceso de desarrollo orientado a objetos. 
\item En el año 2006, se trabajó en temáticas vinculadas a la gestión de proyectos de software y la aplicación de herramientas que 
favorecieran su automatización, poniendo el foco en la mejora de las prácticas vinculadas con la gestión de proyectos de software.
\item Durante el año 2007 se elaboró un proyecto de articulación de contenidos con todas las asignaturas de tercer año. Esto permitió a los 
estudiantes
establecer una relación directa entre los contenidos estudiados en cada una se las asignaturas.
\item A partir del año 2011 se investigaron y analizaron las causas que hacen que un importante número de estudiantes no logren cumplir con la 
planificación temporal establecida para concluir sus proyectos finales de carrera \cite{chino}.
\end{itemize}
El cuerpo docente, autor de este artículo, tiene a cargo el dictado de los cursos Análisis y Diseño de Sistemas e Ingeniería de Software, durante el tercer año de las carreras: Licenciatura y Profesorado en Ciencias de la Computación y Analista en Computación de la  UNRC.
La planificación y ejecución de procesos de enseñanza-aprendizaje para cursos de Ingeniería de Software, plantean un gran desafío a los docentes universitarios involucrados. La necesidad de una actualización dinámica de los contenidos tiene siempre como finalidad la formación integral del estudiante, no solo desde lo académico, 
sino con una fuerte capacitación en las tecnologías utilizadas en la industria.

  
