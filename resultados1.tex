Luego de realizar y analizar  las encuestas, y evaluando el desempeño de los grupos es posible destacar lo siguiente: Todos los
grupos se mostraron muy entusiasmados e interesados con la propuesta. Algunos grupos lograron ampliamente
los objetivos iniciales y otros les costó un poco más lograr adaptarse a las nuevas herramientas.
La modalidad del taller, exige tiempo de investigación fuera del horario de clases. Hubieron grupos que tuvieron un desempeño sobresaliente, obligando al cuerpo docente a proponer nuevas funcionaliades y nuevas herramientas.  
La mayoria de los alumnos desconocían herramientas como Git, JUnit, Pivotal Tracker, ActiveJDBC. Si bien muchas de ellas son mencionadas como ejemplos en contenidos de la asignatura Ingeniería de Software, fue dentro de este proyecto el primer contacto con las mismas. Un gran porcentaje de los grupos expresa que se siente capaz de escalar este proyecto a uno de dimensiones mayores, aplicando las mismas tecnologias. La asignatura Diseño de Algoritmos es la que en general, los alumnos visualizan que tiene menos relación con el proyecto. Esto puede ser debido a la propia naturaleza de la aplicación que se desarrolló. El trabajo en grupo fue muy positivo, la propia naturaleza de herramientas como Git, permite realizar un control a cada integrante del grupo. Si bien, desde el cuerpo docente se realizaba un seguimiento particular de cada alumno, se le otorgaba al grupo autonomía para trabajar. En algunos casos, ellos mismos se dividieron las tareas, por ejemplo cada uno se dedicaba a investigar una herramienta particular para luego mostrarla al resto del grupo. Git fue una de las herramientas
más resistidas en un comienzo. Esto en parte se debe a que Git  establece una nueva modalidad de trabajo, en particular para trabajos distribuidos. En un comienzo provocaba problemas en muchos grupos, hasta que pudieron adoptarla, y finalmente sacar los beneficios  del versionado establecido por Git. 
Con respecto a Scrum, todos los grupos pudieron seguirla sin inconvenientes.  